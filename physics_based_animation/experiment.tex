\documentclass{fancydoc}
\usepackage{amsmath}
\usepackage{amsthm}
\usepackage{hyperref}
\usepackage{todonotes}
%\hypersetup{
%	colorlinks=true,
%	linkcolor=blue,
%	filecolor=magenta,      
%	urlcolor=cyan,
%}

\newtheorem{mydef}{Definition}
\newtheorem{lemma}{Lemma}
\newtheorem{thm}{Theorem}
\newtheorem{note}{Note}
\newtheorem{ex}{Example}

\newcommand{\trans}{\mathrm{T}}

\DeclareMathOperator*{\argmax}{arg\,max}
\DeclareMathOperator*{\argmin}{arg\,min}

\begin{document}

\title{Ladislav Kavan's Physically Based Simulation 2017ed}
\author{Edited by wxgopher}

\maketitle
\tableofcontents
\newpage
{\large This lecture note is based on Dr. Ladislav Kavan's Physics-based Animation course CS6660, originally taught at the University of Utah. I also try to include some supplementary materials from other resources.}


{\large Warning: Although as I strive to make this material useful, there are certain bugs, use this material at your own risk. I would also be grateful to hear \href{mailto:wxguojlu@gmail.com}{feedbacks}.}

\dotfill

\section{Classical Mechanics}

\subsection{Basics: a harmonic oscillator}

\section{Time Integration}

\section{Optimization}
\subsection{Implicit Newmark}

\begin{subequations}
	\begin{align}
	x_{n+1} &= x_n + \frac{h}{2}(v_n + v_{n+1}),\\
	v_{n+1} &= v_n = \frac{h}{2}(f_n + f_{n+1}),
	\end{align}
\end{subequations}
where $x$ and $v$ are positions and velocities, and $f_n \equiv f(x_n)$. Thus we have
\begin{equation}\label{secoptim:nw_eq}
x_{n+1} - x_n - hv_n = \frac{h^2}{4}(f_n + f_{n+1}).
\end{equation}

Let $y = x_n + h_nv_n + \frac{h^2}{4}f_n$, $x_{n+1} = x$, note that $\nabla E = -f$, then \eqref{secoptim:nw_eq} becomes

\begin{equation}
x - y = \frac{h^2}{4}f \Rightarrow x - y + \frac{h^2}{4}f\nabla E = 0.
\end{equation}

Let $g(x) = \frac{1}{2}\Vert x-y\vert^2 + \frac{h^2}{4}E$, then solving $g$ equals to solve $x$ for
\begin{equation}
\min_{x} \quad g(x).
\end{equation}

\subsection{Optimization Problems}
Problem formulation:
\begin{equation}\label{problem_optim_nonlinear}
\min_{_x} \quad g(x), \quad x\in \!R^n, \quad g\in \!R.
\end{equation}

Optimization problems can be categorized into constrained or unconstrained problems, or convex or non-convex problems. 

\begin{thm}
	For a convex problem (where both objective and feasible set are convex), if the objective is $\!C^2$, then the Hessian $H \succeq 0$, and the local minimum is the global minimum.
\end{thm}

\begin{mydef}
	A linear programming (LP) is a problem with linear objective and linear equality or inequality constraints. A quadratic programming (QP) is {\bf the same as LP} except with a quadratic objective.
\end{mydef}

\begin{note}
	Convex QP has polynomial time solver but non-convex QP is NP-hard.
\end{note}
\begin{ex}
	A non-convex QP: 
	\begin{subequations}
		\begin{align}
		\min_{x} \quad&\frac{1}{2} \Vert Ax \Vert^2, \\
		s.t. \quad & \Vert x \Vert_2 = 1.
		\end{align}
	\end{subequations}
\end{ex}
\begin{note}
	Software package for solving non-convex problem: {\rm IpOPT}, {\rm KNITRO}, or \href{https://neos-guide.org/}{NEOS-Guide} 
\end{note}

\subsubsection{Solving unconstrained problems}
There are two ways to solve an unconstrained problem: descent method or trust-region method.

Descent method refers to pick a descent direction $d$ and do an exact or inexact line search~(LS) to determine descent distance $\alpha d$.

\begin{mydef}
	Descent direction: $\forall \alpha \in (0, \alpha_0)$, $\alpha_0 > 0$, $g(x+\alpha d) < g(x)$.
\end{mydef}
\begin{mydef}
	Exact LS refers to solving the following problem:
	\begin{equation*}
	\argmin_{\alpha>0} g(x+\alpha d).
	\end{equation*}
\end{mydef}
\todo[inline]{Backtracking LS}

\subsubsection{Newton's method}
A usual descent direction is called gradient descent~(GD), denoted by $d=-\nabla g$. It doesn't work well sometime, for example:
\begin{ex}
	\begin{equation}
	\min_{x_1, x_2} g(x1, x2) = \frac{1}{2}(x_1^2 + \gamma x_2^2), \quad \gamma > 0.
	\end{equation}
\end{ex}
If we do exact LS, we get
\begin{subequations}
\begin{align}
x_1^k &= \gamma (\frac{\gamma - 1}{\gamma + 1})^k, \\
x_2^k &= (-\frac{\gamma - 1}{\gamma + 1})^k.
\end{align}
\end{subequations}
If gamma becomes huge, we can tell GD converges slowly (as two GD directions are perpendicular).

Newton's method refers to do a quadratic approximation to problem~\eqref{problem_optim_nonlinear}, given by
\begin{equation}
g(x + p) = g(x) + \nabla g^{\trans}(x)p + \frac{1}{2}p^\trans\nabla^2g(x)p + O(\Vert p\Vert^2) \equiv \tilde{g}(x) + O.
\end{equation}
Thus we have
\begin{equation*}
\frac{\partial \tilde{g}}{\partial p} = 0 \Longleftrightarrow \nabla g + \nabla^2 gp = 0.
\end{equation*}
The Newton descent direction is given by
\begin{equation*}
p = -(\nabla^2g)^{-1}\nabla g.
\end{equation*}
\begin{note}
	Newton's method is affine invariant. Let $x=Ty$ with $T$ being a nonsingular matrix, so problem~\eqref{problem_optim_nonlinear} is equivalent to problem 
	\begin{equation}
	\min_{y} \tilde{g}(y).
	\end{equation}
	We also have 
	\begin{equation*}
	\nabla \tilde{g} = T^2 \nabla g, \quad \nabla^2 \tilde{g} = T^
	\trans \nabla^2 g T.
	\end{equation*}
	Newton direction $\Delta y$ of $\tilde{g}$ and $\Delta x$ of $g$ is given by
	\begin{equation*}
	\Delta y = -T^{-1}\nabla^2 g \nabla g = T^{-1} \Delta x.
	\end{equation*}
\end{note}


\subsection{Numerical Linear Algebra}

\section{Elastic Materials and Finite Element Simulation}

\section{Something more...}

\section{Papers and books}
\begin{enumerate}
	\item Nonlinear Continuum Mechanics for Finite Element Analysis by J. Bonet
	\item Convex optimization by S. Boyd
	\item Numerical optimization by Nocedal
\end{enumerate}

\end{document}